\def\Titel{Drei glänzende Kugeln}
\def\Interpret{Franz Josef Degenhard}
\def\Referenz{text}

\LiedSetup{}

\begin{guitarMagic}
    \begin{enumerate}
        \item Es [Am]liegen drei [E]glänzende [Am]Kugeln\\
            Ich [Dm]weiß nicht, wo[E]raus ge[Am]macht\\
            In [Am]einer [E]niedrigen [Am]Kneipe\\
            Neun [Dm]Meilen [E]hinter der [Am]Nacht!\\
            Sie [E]liegen auf grünem [Am]Tuch\\
            und [E]an der Wand hängt der [E7]Spruch:\\
            \ \\
            [Refrain]\\
            [F]Wer die [G]Kugeln [C]rollen [Am]lässt\\
            [Dm]Darada-[G]diri[C]dum\\
            [F]Den über[G]kömme die [C]schwarze [Am]Pest\\
            [E]Tralala-[E7]diri[Am]dum!\\


        \item Der Wirt, der hat nur ein Auge\\
            Und das trägt er hinter dem Ohr\\
            \liedweiter
            Aus seinem gespaltenen Kopfe\\
            Ragt eine Antenne hervor\\
            Er trinkt aus einer Seele\\
            Und ruft aus roter Kehle:

        \item Die Einen sagen, die Kugeln\\
            Sind die Sonne, die Erde, der Mond\\
            Die Andern glauben, sie seien\\
            Das Feuer, die Angst und der Tod\\
            Und wenn sie beisammen sind\\
            Dann summen sie in den Wind:

        \item Und dann kam einer geritten\\
            Es war in dem Jahr vor der Zeit\\
            Auf einer gesattelten Wolke\\
            Von hinter der Ewigkeit!\\
            Er nahm von der Hand einen Queue\\
            Der Wirt rief krächzend: He!

        \item Doch jener, der lachte zwei Donner\\
            Und wachste den knöchernen Stab\\
            Visierte und stieß, und die Kugeln\\
            sie rollten, der Wirt grub ein Grab!\\
            Fäulnis flatterte auf\\
            So nahm alles seinen Lauf:
    \end{enumerate}
\end{guitarMagic}
